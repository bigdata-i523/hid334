\documentclass[sigconf]{acmart}

\usepackage{graphicx}
\usepackage{hyperref}
\usepackage{todonotes}

\usepackage{endfloat}
\renewcommand{\efloatseparator}{\mbox{}} % no new page between figures

\usepackage{booktabs} % For formal tables

\settopmatter{printacmref=false} % Removes citation information below abstract
\renewcommand\footnotetextcopyrightpermission[1]{} % removes footnote with conference information in first column
\pagestyle{plain} % removes running headers

\newcommand{\TODO}[1]{\todo[inline]{#1}}
\newcommand{\DONE}[1]{DONE: \todo[inline,color=green!30]{#1}}

\begin{document}
\title{Advancements in Drone Technology for the U.S. Military}

\author{Peter Russell}
\affiliation{%
  \institution{Indiana University}
}
\email{petrusse@iu.edu}

\begin{abstract}
Technological breakthroughs in military technology have put the U.S. in a new a chapter of warfare. These advancements have become realizations of what was at one time only deemed possible in science fiction, such as autonomous decision making and weaponization of drones. These innovations provide unique advantages to the leaders of the technology and are too lucrative to ignore. However, in coming years as these technologies push the boundaries, decisions will need to be made in how much control military leaders are willing to give to their new mechanic allies and whether they should be passive, as they have been in the past, or active participants on the battlefield.
\end{abstract}

\keywords{i523, Drone Technology, Big Data, Military Technology, Big Data}

\maketitle

\section{Introduction}
Among the many industries being transformed by the Big Data movement, few carry more consequences than the changes being experienced in the U.S. military. It has been argued that the current changes could affect warfare and diplomatic landscape on the same scale that nuclear weapons did \cite{aireport}. 

Traditionally, drones can be categorized as a re-usable, autonomous vehicle either in the air or on the ground. In the air, these are known as ``Unmanned Aerial Vehicles", or UAVs, and on the ground, ``Unmanned Ground Vehicles", or UGVs. More recently, this has expanded to USVs for ``Unmanned Surface Vehicles" and UUVs for ``Unmanned Underwater Vehicles." Our focus will remain primarily on the former two types, UAVs and UGVs since these have been the most long-standing types.

For most citizens, UAVs remain the more well known of the two and have garnered more attention recently for their proposed commercial and consumer uses. Unsurprisingly, this has created rapid growth in the industry. In 2017 alone, the $\$6$ billion industry for these uses is expected to grow by $35\%$, roughly matching its 2016 growth \cite{gartner}. While these market segments are growing quickly, they remain in their nascent stages and dwarfed by drone spending in the Department of Defense (DoD). To put the difference of size in perspective, for the FY2018 budget, the DoD requested nearly $\$7$ billion \emph{for the year} in drone spending, which is the highest since FY2013 and $\$3.3$ billion than previously estimated four years ago \cite{dronebudget}. This annual spending is no anomaly. Currently, the DoD is responsible for nearly $90\%$ of the spending in the UAV market \cite{economist}. 

This continued investment in the drone program demonstrates a clear vote of confidence in the advancement of this technology and its impact on military operations. 
%Given the massive DoD budget of $\$895$ billion, which is the highest of any country and more than the next nine countries combined \cite{forbesdrone}.

%The DoD has made a steady investment into drone technology every year with the area comprising about $3\%$ of the total budget annually. 


\section{Strategic Advantage with Drones}
One facet in the complexity of military planning is finding the path that gives the highest probability of success with the lowest possible risk of casualties. In this light, the stakes of the problem that technology is trying to solve with Big Data could not be higher. Leaders are constantly trying to solve a constrained optimization problem and when it comes to this overarching problem of mission success, Big Data can be utilized to help the decision maker solve the smaller sub-problems that comprise it, such as where to set up surveillance, where and when to pursue the enemy, how to carry out reconnaissance and how to disarm hazardous obstacles along the way.

\subsection{Surveillance}
The RQ-4 Global Hawk is currently America's most expensive surveillance drone, projected to cost nearly $\$428$ million in 2018 \cite{dronebudget}. Having flown missions for nearly 15 years now, the total cost of this drone is over $\$14$ billion \cite{rq4}. This drone provides a great case study in the evolution of military drone capabilities for its long track record, which stands at over 200,000 flight hours \cite{northrq4}. 

Initially, the RQ-4 provided imagery through its equipped sensors, which were for landscape topography (synthetic aperature radar), navigation (electro-optical sensors) and heat signatures (infrared sensors). Later models have been equipped with features that allow the antennae to move on its own for an improved signal and a radar tracking system that allows its operators to zero in a target from its surroundings (moving target indication). 

The advancement of sensors provides leaders with high resolution photos for strategic planning. In its current capability, images can be obtained with up to a 1 foot resolution and targeting precision within a 60 foot radius from its maximum height of 65,000 feet \cite{forbesdrone}. However, it should be noted, the revolutionary aspect in drones does not necessarily come entirely from its sensors. The U-2, which was first introduced in 1955 and famously downed over the Russian border during the Cold War, can be retrofitted with these senors. The fact that these modern planes can be unmanned, thereby avoiding a diplomatic crisis in the worst case scenario as was the case with the U-2, and sustain flights in excess of 30 hours \cite{foxtrot}, exceeding the limits of human focus and endurance. 

\subsection{Swarms}
One of the most exciting applications of drone technology revolves around drone swarms. Spending in this category, broadly defined as ``Autonomy, Teaming and Swarms'' has doubled in the last four years \cite{dronebudget}. With UGV spending stagnant over this period, this program is now receiving twice as much funding, but is still only a small fraction of the largest program, unmanned aircrafts, at $10\%$ of that spending. The rapid growth in this program will undoubtedly continue given the revolutionary nature of these swarms as it relates to warfare.

The public has been aware of this new technology since early 2016, but its development has been underway since at least 2014 \cite{washpost}. The program, however, did not gain mass attention until a 60 Minutes special aired in early 2017 introducing the Perdix drone and demonstrating a mock swarm mission comprised of 103 Perdix drones acting as a single unit \cite{60minutes}. 

The Perdix drone looks similar to a toy airplane, weighing only a pound and with a wingspan of 6.5 inches. This simple design reflects the expendability of each drone, which is one of the swarms major advantages. In the swarm, there is no lead drone in the swarm and therefore, no single vulnerability to attack if one of the drones was taken down by the enemy \cite{mitswarm}. As a result, each drone is designed to work with other drones of the same type as a single unit to achieve a given mission objective and fill in any gaps if drones are no longer functional. These drone swarms are intended to be able to scan large areas very quickly, provide electronic jamming against the enemy, create a wide communication area for ground troops or confuse enemy radar \cite{popularmechanics}.

In the 60 Minutes demonstration, these Perdix drones were dropped from F-18 jets at the speed of sound, aggregated together and collectively scanned an area, entirely on their own. The innovation in computing and Big Data allows the swarm to exist as no human individually or as a team could make the calculations that these drones are making collectively as one unit to achieve their mission.

 At the moment, while also being a means of surveillance like the RQ-4, these swarms are not a replacement for these traditional drones, nor does that seem to be the end goal. These Perdix drones have a flight time of only 20 minutes currently and flown at a relatively low-altitude. This compares with with the RQ-4, which is considered a HALE, or High Altitude Long Endurance drone. Additionally, the RQ-4 requires a team of nearly a dozen while the swarm is given a directive by an operator on its objective and requires no human intervention \cite{ftswarm}. Lastly, with a unit cost of $\$235$ million per unit, the RQ-4 holds an economic liability with any enemy attack that the Perdix does not at only $\$30,000$ per unit. 

Eventually, these swarm drones are expected to have the capability to be aggregated together by the thousands and carry out overwhelming and confusing attacks on enemies. It has been properly described as the ``difference between a wolf pack and just little wolves\cite{ftswarm}."

\subsection{Disarmament and Detection}
Of these two drone segments, UAVs remain by far the larger of the two with spending on UAVs nearly 20x that of UGVs\cite{dronebudget}. To date, UGVs have been responsible for aiding ground troops in their mission. While this could come in the form of reconnaissance or in helping carry heavy loads, explosive detection has arguably been the most important impact for their ability to screen areas for improvised explosive devices (IEDs) along paths that ground troops must travel to complete their mission. 

In comparison to aerial innovations, UGV development has developed at a slower pace when it comes to full autonomy. This is largely due to the nature of challenges a ground drone faces in navigation versus flying. Namely, how to deal with uneven terrain and unpredictable obstacles \cite{nytugv}. Nonetheless, user operated UGVs have proven to be a tremendous advantage as it relates to disarmament and detection.

To circumvent the endless and unique possible situations a UGV could be faced with, the military been innovative in the way these UGVs are deployed instead to avoid these hurdles. For example, soldiers can now throw a five pound UGV from a height of up to 15 feet to begin a reconnaissance or bomb detection mission \cite{throwugv}. This allows them to be thrown on top of a roof or into openings that humans might not be able to fit. These robots are equipped with video cameras and various sensors to relay information about the landscape back to the operator.  

\section{Recent Developments}
In the field of surveillance, one of the newest drones being pursued is the Zephyr 8, a solar powered drone that can fly for 45 \emph{days} continuously. This flight time allows the drone to be launched in the U.S. and reach destinations like Afghanistan on its own, but perhaps even more incredible is the amount of data this drone can produce. Specifically, it flies at a height of nearly 12.5 miles in the sky, far exceeding the height needed to see the curvature of the Earth, but can still take pictures at the precision of 6-inch resolution. This height allows surveillance of 386 square miles and coupled with this resolution, this becomes a large data set very quickly \cite{foxdrone}. 
One of the newest developments in drone technology by the U.S. military does not categorize as a UAV or UGV, but instead as a USV, for Unmanned Surface Vehicle. These are autonomous boats with the most famous example to date being the Sea Hunter, which was introduced in 2016. This massive vessel, with the length of 132 feet and 135 tons, was built to track diesel submarines and detect mines \cite{seahunter}. It is a major innovation for the U.S. military for its range, which is 12,000 miles on a single tank of gas, and its economic savings, which is $2\%$ of what a traditional ship costs to operate daily. \cite{seahuntergas} \cite{seahuntercost}. Or, framed differently, the U.S. military can operate 50 of these Sea Hunter ships for the same cost as one traditional ship. This has proven to be a Big Data and computational marvel as the ship operates autonomously through 36 computers running 50 million lines of code \cite{60minutes}.

\section{Future Developments}
One of the aspirational areas of future drone development for the military is in the field of Micro Air Vehicles (MAVs), which as the name implies, are extremely small UAVs, such as the size of a small bird. Even within that area, there is a growing interest in Nano Air Vehicles, which could be the size of an insect. The future of this technology is for troops to gain intelligence on areas that would be too dangerous to enter or physically impossible. 

One of the more well-known MAVs is the Black Hornet Nano. The drone measures 4 inches in length and is an inch wide with the weight of just a half an ounce, or the weight of 3 pieces of paper. This drone has three cameras and can fly for 20 minutes non-stop. Interestingly, the drone is designed to stream video back to its operators to avoid the risk of footage being compromised if it were stored locally. While this is all extremely impressive, future developments are pushing to make these MAVs even smaller. However, the smaller and lighter these MAVs become, the harder they become for a user to control. The reason being is that the smaller they are, the more sensitive they become to cross winds, the more difficult they are to equip with navigation sensors and the smaller field of vision the camera has. However, the inability to be detected by enemies is a tremendous advantage and to circumvent these piloting issues, work is being done to make them fully autonomous, potentially even as a swarm. 

\section{Integration of Drone Technologies}

One of the beautiful aspects of technological innovation are the synergies created. For the U.S. military, these synergies in the context of the Big Data movement are opening new possibilities with difficult questions that will eventually have to be answered. An example of this is how or if drones should be weaponized, even if their decision making is superior to humans. 

Without Big Data this debate could never take place. For example, one of the highest resolution drone surveillance cameras in 2014, ARGUS-IS, disclosed some, but not all, of its features as some parts remained classified. It was equipped with a 1.8 billion megapixel camera that could monitor 10 square miles and store all of this information, which works out to be 6 petabytes of data daily \cite{argus}. 

This information accumulation allows greater monitoring of potential targets. Namely, if a known target is tagged and tracked, pictures can be taken at different angles and stored in a database. This ability to accumulate a massive amount of data improves the accuracy of facial recognition. One demonstration showed how a low-altitude drone could be coupled with a UGV and USV against an enemy. The low-altitude and UGV would work in conjunction with each other to carry out a reconnaissance and scan an area and once a match of the target has been found, communicate this to the USV in a different location to fire the weapon systems on the target \cite{60minutes}. 

While this chain of events is currently possible, which is to attack an enemy with no human interaction, there is a difficult ethical choice to be made in how, or if, these drones will be integrated with respect to weaponization. Even if computers are able to make better decisions on facial recognition, which recent evidence suggests that they can, there remains a large reluctance to open this potential Pandora's box as a new type of warfare \cite{googleface}. 

\section{Conclusion}
Adoption of drone and autonomous technology has become the modern arms race and the U.S. has shown itself willing to push to the forefront of these new technologies. This new arms race is unlike the nuclear arms race in that there is no clear first mover, or innovator, advantage. Instead, in the era of Big Data, as shown in the use example of these technologies, the operator that is best able to use the vast amount of information available to them simultaneously will hold the advantage. The U.S. is making promising steps towards this end and will face new, difficult choices in how to integrate these innovations.

%This innovation, of course, creates a cumulative effect whereby more data gives way to more insight and therefore, new solutions or innovation. 

%The paper should be 2 written pages excluding figures and references. For %your chosen topic, your paper should answer the following questions:
%Why is this topic important?
%How is it relevant to Big Data?
 
%More information on scope:
%You should assume not much knowledge, common knowledge is ok, but as you %have only 2 pages you need to make sure you address
 
%a) what is the problem
%b) why is big data involved
%c) how can big data or analytics of big data help
%d) what infrastructure/programs/systems exist for this
 

These recent uses have in many ways eclipsed the previous association the majority of citizens have had with drones. For most, their introduction to drones came through the usage in a military context at the beginning of the War on Terrorism in 2001. However, drones utilization in the military began far earlier, as will be shown. Additionally, while these commercial and personal have captured recent interest, military drones still comprise $90\%$ of the market and have been one of the largest beneficiaries of the Big Data revolution.


\begin{acks}
The author would like to thank Dr. Gregor von Laszewski and the Associate Instructors for their support and suggestions in exploring this topic.
\end{acks}

\bibliographystyle{ACM-Reference-Format}
\bibliography{report} 

%\section{Issues}

\DONE{Example of done item: Once you fix an item, change TODO to DONE}

\subsection{Assignment Submission Issues}

    \TODO{Do not make changes to your paper during grading, when your repository should be frozen.}

\subsection{Uncaught Bibliography Errors}

    \TODO{Missing bibliography file generated by JabRef}
    \TODO{Bibtex labels cannot have any spaces, \_ or \& in it}
    \TODO{Citations in text showing as [?]: this means either your report.bib is not up-to-date or there is a spelling error in the label of the item you want to cite, either in report.bib or in report.tex}

\subsection{Formatting}

    \TODO{Incorrect number of keywords or HID and i523 not included in the keywords}
    \TODO{Other formatting issues}

\subsection{Writing Errors}

    \TODO{Errors in title, e.g. capitalization}
    \TODO{Spelling errors}
    \TODO{Are you using {\em a} and {\em the} properly?}
    \TODO{Do not use phrases such as {\em shown in the Figure below}. Instead, use {\em as shown in Figure 3}, when referring to the 3rd figure}
    \TODO{Do not use the word {\em I} instead use {\em we} even if you are the sole author}
    \TODO{Do not use the phrase {\em In this paper/report we show} instead use {\em We show}. It is not important if this is a paper or a report and does not need to be mentioned}
    \TODO{If you want to say {\em and} do not use {\em \&} but use the word {\em and}}
    \TODO{Use a space after . , : }
    \TODO{When using a section command, the section title is not written in all-caps as format does this for you}\begin{verbatim}\section{Introduction} and NOT \section{INTRODUCTION} \end{verbatim}

\subsection{Citation Issues and Plagiarism}

    \TODO{It is your responsibility to make sure no plagiarism occurs. The instructions and resources were given in the class}
    \TODO{Claims made without citations provided}
    \TODO{Need to paraphrase long quotations (whole sentences or longer)}
    \TODO{Need to quote directly cited material}

\subsection{Character Errors}

    \TODO{Erroneous use of quotation marks, i.e. use ``quotes'' , instead of " "}
    \TODO{To emphasize a word, use {\em emphasize} and not ``quote''}
    \TODO{When using the characters \& \# \% \_  put a backslash before them so that they show up correctly}
    \TODO{Pasting and copying from the Web often results in non-ASCII characters to be used in your text, please remove them and replace accordingly. This is the case for quotes, dashes and all the other special characters.}
    \TODO{If you see a figure and not a figure in text you copied from a text that has the fi combined as a single character}

\subsection{Structural Issues}

    \TODO{Acknowledgement section missing}
    \TODO{Incorrect README file}
    \TODO{In case of a class and if you do a multi-author paper, you need to add an appendix describing who did what in the paper}
    \TODO{The paper has less than 2 pages of text, i.e. excluding images, tables and figures}
    \TODO{The paper has more than 6 pages of text, i.e. excluding images, tables and figures}
    \TODO{Do not artificially inflate your paper if you are below the page limit}

\subsection{Details about the Figures and Tables}

    \TODO{Capitalization errors in referring to captions, e.g. Figure 1, Table 2}
    \TODO{Do use {\em label} and {\em ref} to automatically create figure numbers}
    \TODO{Wrong placement of figure caption. They should be on the bottom of the figure}
    \TODO{Wrong placement of table caption. They should be on the top of the table}
    \TODO{Images submitted incorrectly. They should be in native format, e.g. .graffle, .pptx, .png, .jpg}
    \TODO{Do not submit eps images. Instead, convert them to PDF}

    \TODO{The image files must be in a single directory named "images"}
    \TODO{In case there is a powerpoint in the submission, the image must be exported as PDF}
    \TODO{Make the figures large enough so we can read the details. If needed make the figure over two columns}
    \TODO{Do not worry about the figure placement if they are at a different location than you think. Figures are allowed to float. For this class, you should place all figures at the end of the report.}
    \TODO{In case you copied a figure from another paper you need to ask for copyright permission. In case of a class paper, you must include a reference to the original in the caption}
    \TODO{Remove any figure that is not referred to explicitly in the text (As shown in Figure ..)}
    \TODO{Do not use textwidth as a parameter for includegraphics}
    \TODO{Figures should be reasonably sized and often you just need to
  add columnwidth} e.g. \begin{verbatim}/includegraphics[width=\columnwidth]{images/myimage.pdf}\end{verbatim}

re


\end{document}
