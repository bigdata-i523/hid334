\documentclass[sigconf]{acmart}

\usepackage{graphicx}
\usepackage{hyperref}
\usepackage{todonotes}

\usepackage{endfloat}
\renewcommand{\efloatseparator}{\mbox{}} % no new page between figures

\usepackage{booktabs} % For formal tables

\settopmatter{printacmref=false} % Removes citation information below abstract
\renewcommand\footnotetextcopyrightpermission[1]{} % removes footnote with conference information in first column
\pagestyle{plain} % removes running headers

\newcommand{\TODO}[1]{\todo[inline]{#1}}
\newcommand{\DONE}[1]{DONE: \todo[inline,color=green!30]{#1}}

\begin{document}
\title{Apps: At The Intersection of Big Data and IoT }

\author{Peter Russell}
\affiliation{%
  \institution{Indiana University}
}
\email{petrusse@iu.edu}

\begin{abstract}
Test
\end{abstract}

\keywords{i523, HID 334, Edge Computing, Raspberry Pi, IoT}

\maketitle


\section{Introduction}
In 2020, it is estimated that 95 percent of electronics will contain IoT technology. This type of technology, with IoT being short for ``Internet of Things'', is broadly defined in its application, which could come in the form of a phone, vehicle, a home device like a thermostat or television, but rather specific in its intention. Put simply, IoT technology is intended to describe devices that collect and relay information via the Internet. They are different than a computer though in that an IoT device is built to serve a specific purpose or job, but not do the actual heavy computing itself. Nevertheless, computing developments have been central to the recent growth in the IoT. Innovations in computing power and speed have increased the amount of data that can be collected and processed, typically referred to ``Big Data'', which has enabled IoT devices to become more personalized and useful. The utility of these devices has spurred tremendous growth recently, on the order of $30 \%$ annually and 2017 is expected to be the year that the number of IoT devices exceeds the number of people on Earth. %https://www.gartner.com/newsroom/id/3598917

As the growth in IoT, aided by Big Data, continues in coming years, scalability of the devices becomes a central concern. Further growth from this point means multiple IoT devices per person, which means an increasing number of providers will need to find a solution to this problem. 

\section{IoT Project}
\subsection{Description}
The project undertaken was to design an application that demonstrates how an IoT device can be used to integrate Big Data analytics for a desired output. In this case, the goal was to create a personalized interface that gives the user a morning snapshot for relevant, important information to begin their day. 

\subsection{Implementation}
The application was developed using Python, utilizing the Kivy package for GUI development, the requests and Beautiful Soup packages for the user location, news stories and sports scores along with the  Yahoo Weather/Weather Underground via the Weather package. 

\subsection{Results}
As part of the display, a continuous running clock was added, which necessitated the application to be on a constant refresh. This was implemented successfully and at a relatively low cost with no significant delays. The total build of the application consumed 966,565 bytes with each refresh using just 1032 bytes. 
At the initialization of the program, the application uses the user's IP address to find their zip code to populate the local weather forecast and local news. For the weather, the user will get the current temperature with a high/low for the current day and each day in the five day forecast.

Additionally, as news stories are published to the WSJ news feed, they will be read into the application and refreshed. Stories are shown in chronological order along with their time stamp of publication and each is hyperlinked directly to the full story if the user wants more information. 

\subsection{Application to Big Data}
One of the main benefits of IoT is synthesizing data across numerous platforms and data sources into a desired output. As is likely evident by the description of the project and will be expanded on further, this application benefits greatly from the creativity by the providers in solving difficult Big Data issues, such as the  clustering of news (Google) and weather forecasting (Yahoo/Weather Underground). 

\subsection{Google News}
\subsubsection{Big Data Description}
Google News has evolved into a central source of information for how a large share of the population receives its news. In fact, as a display of the trust that users place in Google to deliver the most information in the most efficient manner, it was found that users are more likely to trust a Google news headline than that same headline from the original source \cite{edelman}. Additionally, $44\%$ of users were found to read nothing more than the headlines \cite{googleheadlines}. This is a testament to their ability to simplify the universe of world news into succinct rankings. 

Entering into its fifteenth year, Google News aggregates from 50,000 news sources worldwide across 30 different languages. In 2012, they reported the division was receiving 1 billion unique visits a \emph{week}\cite{krishna}. For reference, major individual news providers, such as CNN and the New York Times receive 125 million and 99 million unique visitors per \emph{month} \cite{nytimes}. These statistics further demonstrate Google's successful navigation of the Big Data problem for news stories in the eyes of its users. 
It's relatively clear why this is an important Big Data problem, but one might be curious how they're able to effectively navigate the problem. Unfortunately, the full design from start to finish is a well kept secret, but pieces have been released and one can piece together a mosaic view of what might be going on under the hood. The decision to not disclose the techniques for ranking news stories is understandable, but it has been a lightening rod of controversy nonetheless. Some view the decision of Google's scoring as effectively acting as a censor for the internet while they maintain it is to keep the integrity of the algorithm so that news stories cannot be written purely for traffic, known as search engine optimization \cite{censor}. 

On the surface, one might question the economic value of Google News to the larger company since it is a free service for both users and providers. However, there is a tremendous amount to be gained in solving this Big Data problem. Even though Google does not show ads on its news site, it was estimated to generate spillover traffic into its search engine that leaves the News entity worth $\$100$ million in 2008 \cite{newsvalue}. The current valuation is undoubtedly higher. So while there are profits to be made for Google in this quest, publishers of these stories have a tremendous amount of interest in this problem as well. Some providers don't believe content should be indexed to Google's search algorithm for free and Google should pay them for their investigative research. One such provider, Axel Springer, Germany's largest news source decided to remove themselves from the index for two weeks and the results were devastating for the site as traffic through the site dropped by $40\%$ \cite{springer}.

\subsubsection{Big Data Solution}
While Google News has undergone many innovations, especially in recent years with demand by users to auto-detect fake news, the information that has been released is relevant, but dated. Of what has been uncovered, the primary algorithms in the early versions used in this Big Data problem were MinHash, Probabilistic Latent Semantic Indexing and Covisitation. Specifically, these methods will compare historical clicks with other similar users for recommendations, decipher key words and phrases from an article for grouping and track how news stories are clicked within a certain time frame to find stories that were read successively.  

For processing these queries, Google uses MapReduce and Hadoop architecture \cite{googlearch}.  
\subsection{Weather Underground via Yahoo API}
\subsubsection{Big Data Description}
Weather is a primary concern in planning for many businesses and as a result, companies are willing to dedicate a tremendous amount of resources towards accurate forecasts. One of the most innovative companies and a great example of the intersection of IoT and Big Data is Weather Underground. 

Weather Underground is a weather forecasting service that was once owned by the Weather Channel and recently, partially by IBM to integrate with its growing IoT ecosystem. What makes the company unique is how their forecasts are formulated. In their model, forecasts provided by the National Weather Service (NWS), which aggregates data from airports and weather balloons, are pooled along with data from personalized weather stations that are maintained by its users, which number over 250,000. This provides an additional layer of information, yielding more frequent data, longer forecast windows and greater certainty for a given area. Namely, users can get new forecasts every 15 minutes (versus every 4 hours on the NWS) and forecasts up to two weeks in advance (compared to one week for NWS) \cite{wuabt}. This use of the IoT, specifically edge computing, which will be expanded on later, provides a tremendous example of how IoT can be used to enhance Big Data analysis. 

For those that choose to participate in the service, they will purchase a Personal Weather Station (PWS) that allows them to measure temperature, humidity, pressure, rainfall, wind speed and direction via sensors. The major advantage of the PWS comes from its pressure and wind metrics as users can get a better idea of humidity and wind chill, giving a more accurate representation of current conditions. Neither of these are available through the NWS. In the end, this amounts to around 3 billion data points for the Weather Underground model, servicing around 26 billion inquiries a day \cite{ibm}.

\subsubsection{Big Data Solution}
To process its data in the past, which amounts to multiple terabytes daily, Weather Underground has stored its forecasts, radar data and satellite data using Apache Hadoop and Amazon Web Services \cite{wuinf}. In fact, IBM has stated a large reason for their motivation to have an ownership stake in the company was due to the cloud infrastructure that Weather Underground had built for fielding the massive volume of requests and forecasts it processes daily.  

\subsection{Edge Computing}
With the Personal Weather Systems, we were able to see how IoT can complement Big Data analysis. This is an example of an emerging technology known as ``edge computing'' that is transforming how the cloud is utilized. 

Edge computing gains its name from how the information being processed by the device.  Prior to this recent innovation, information was gathered, sent to the cloud, processed there, and then the output is pushed back to the device. Namely, it was a centralized process. However, with edge computing, devices are more intelligent in what information they choose to send, providing a much more efficient process. For example, rather than having a camera monitor an area constantly, even when there is no motion, modern IoT cameras have been equipped with motion detection so information is only sent when there is something to actually record. Since this decision and processing is made on the actual device, it is considered to be at the \emph{edge} of the network.

Traditionally, individual devices that were intended to work in conjunction, such as surveillance cameras, were simple in their functionality and storage. Namely, a group of cameras would record individually and send their results back to a central server. However, with improvements in image quality, this can become a Big Data problem very quickly as these cameras are running around the clock collecting footage. In the historical model of a centralized server, this setup eventually creates problems as bandwith and storage issues emerge. These limitations are the problems that edge computing seeks to circumvent and has become a major catalyst in the growth of IoT devices \cite{edge}.   

Circling back to the original project that was undertaken, the application benefited from edge computing through the weather data, but the device itself serves as a great example of why edge computing is even possible in the first place. Specifically, it is possible due to the dramatic decrease in computing costs. For the cost of $\$10$ one can get a single-board computer with 1 GHz and 512 MB RAM through the Raspberry Pi. This type of processing is close to becoming the majority as it is expected that by 2019, $45\%$ of all data collected by IoT devices will be processed at the edge of the network \cite{msft}. 





%Google news factors: https://www.theguardian.com/technology/2013/feb/25/1

%https://techcrunch.com/2017/08/03/edge-computing-could-push-the-cloud-to-the-fringe/




%The paper should be 2 written pages excluding figures and references. For %your chosen topic, your paper should answer the following questions:
%Why is this topic important?
%How is it relevant to Big Data?
 
%More information on scope:
%You should assume not much knowledge, common knowledge is ok, but as you %have only 2 pages you need to make sure you address
 
%a) what is the problem
%b) why is big data involved
%c) how can big data or analytics of big data help
%d) what infrastructure/programs/systems exist for this
 
\begin{acks}
The author would like to thank Professor Dr. Gregor von Laszewski, Juliette Zerick and the other Associate Instructors for their support and suggestions in exploring this topic.
\end{acks}

\bibliographystyle{ACM-Reference-Format}
\bibliography{report} 

%\section{Issues}

\DONE{Example of done item: Once you fix an item, change TODO to DONE}

\subsection{Assignment Submission Issues}

    \TODO{Do not make changes to your paper during grading, when your repository should be frozen.}

\subsection{Uncaught Bibliography Errors}

    \TODO{Missing bibliography file generated by JabRef}
    \TODO{Bibtex labels cannot have any spaces, \_ or \& in it}
    \TODO{Citations in text showing as [?]: this means either your report.bib is not up-to-date or there is a spelling error in the label of the item you want to cite, either in report.bib or in report.tex}

\subsection{Formatting}

    \TODO{Incorrect number of keywords or HID and i523 not included in the keywords}
    \TODO{Other formatting issues}

\subsection{Writing Errors}

    \TODO{Errors in title, e.g. capitalization}
    \TODO{Spelling errors}
    \TODO{Are you using {\em a} and {\em the} properly?}
    \TODO{Do not use phrases such as {\em shown in the Figure below}. Instead, use {\em as shown in Figure 3}, when referring to the 3rd figure}
    \TODO{Do not use the word {\em I} instead use {\em we} even if you are the sole author}
    \TODO{Do not use the phrase {\em In this paper/report we show} instead use {\em We show}. It is not important if this is a paper or a report and does not need to be mentioned}
    \TODO{If you want to say {\em and} do not use {\em \&} but use the word {\em and}}
    \TODO{Use a space after . , : }
    \TODO{When using a section command, the section title is not written in all-caps as format does this for you}\begin{verbatim}\section{Introduction} and NOT \section{INTRODUCTION} \end{verbatim}

\subsection{Citation Issues and Plagiarism}

    \TODO{It is your responsibility to make sure no plagiarism occurs. The instructions and resources were given in the class}
    \TODO{Claims made without citations provided}
    \TODO{Need to paraphrase long quotations (whole sentences or longer)}
    \TODO{Need to quote directly cited material}

\subsection{Character Errors}

    \TODO{Erroneous use of quotation marks, i.e. use ``quotes'' , instead of " "}
    \TODO{To emphasize a word, use {\em emphasize} and not ``quote''}
    \TODO{When using the characters \& \# \% \_  put a backslash before them so that they show up correctly}
    \TODO{Pasting and copying from the Web often results in non-ASCII characters to be used in your text, please remove them and replace accordingly. This is the case for quotes, dashes and all the other special characters.}
    \TODO{If you see a figure and not a figure in text you copied from a text that has the fi combined as a single character}

\subsection{Structural Issues}

    \TODO{Acknowledgement section missing}
    \TODO{Incorrect README file}
    \TODO{In case of a class and if you do a multi-author paper, you need to add an appendix describing who did what in the paper}
    \TODO{The paper has less than 2 pages of text, i.e. excluding images, tables and figures}
    \TODO{The paper has more than 6 pages of text, i.e. excluding images, tables and figures}
    \TODO{Do not artificially inflate your paper if you are below the page limit}

\subsection{Details about the Figures and Tables}

    \TODO{Capitalization errors in referring to captions, e.g. Figure 1, Table 2}
    \TODO{Do use {\em label} and {\em ref} to automatically create figure numbers}
    \TODO{Wrong placement of figure caption. They should be on the bottom of the figure}
    \TODO{Wrong placement of table caption. They should be on the top of the table}
    \TODO{Images submitted incorrectly. They should be in native format, e.g. .graffle, .pptx, .png, .jpg}
    \TODO{Do not submit eps images. Instead, convert them to PDF}

    \TODO{The image files must be in a single directory named "images"}
    \TODO{In case there is a powerpoint in the submission, the image must be exported as PDF}
    \TODO{Make the figures large enough so we can read the details. If needed make the figure over two columns}
    \TODO{Do not worry about the figure placement if they are at a different location than you think. Figures are allowed to float. For this class, you should place all figures at the end of the report.}
    \TODO{In case you copied a figure from another paper you need to ask for copyright permission. In case of a class paper, you must include a reference to the original in the caption}
    \TODO{Remove any figure that is not referred to explicitly in the text (As shown in Figure ..)}
    \TODO{Do not use textwidth as a parameter for includegraphics}
    \TODO{Figures should be reasonably sized and often you just need to
  add columnwidth} e.g. \begin{verbatim}/includegraphics[width=\columnwidth]{images/myimage.pdf}\end{verbatim}

re


\end{document}
